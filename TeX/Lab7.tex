\documentclass{article}
\usepackage[height=25cm, a4paper, hmargin={3cm, 2cm}]{geometry}
\usepackage[T1,T2A]{fontenc}
\usepackage[utf8]{inputenc}
\usepackage[english,russian]{babel}
\usepackage[section]{placeins}
\usepackage{graphicx}
\usepackage{lscape}
\usepackage{pdflscape}
\usepackage{minted}
\usepackage{amsmath}
\usepackage[numbered, framed]{matlab-prettifier}
\graphicspath{{}}

\makeatletter
\long\def\@makecaption#1#2{%
  \vskip\abovecaptionskip
  \hbox to\textwidth{\hfill\parbox{0.8\textwidth}{\begin{center}#1: #2\end{center}}\hfill}
  \vskip\belowcaptionskip}
\makeatother

\begin{document}
	\pagestyle{empty}
	\begin{titlepage}
	\begin{center}
		\large{
		Министерство образования и науки Российской Федерации\\Федеральное государственное автономное образовательное учреждение высшего образования}\\
		\vspace*{3em plus 7em minus 6em}
		\Large
		САНКТ-ПЕТЕРБУРГСКИЙ НАЦИОНАЛЬНЫЙ ИССЛЕДОВАТЕЛЬСКИЙ \\УНИВЕРСИТЕТ ИНФОРМАЦИОННЫХ ТЕХНОЛОГИЙ, МЕХАНИКИ И ОПТИКИ\\
		\vspace*{3em plus 7em minus 6em}
		\Large
		Факультет систем управления и робототехники\\
		\vspace*{7em plus 7em minus 6em}
		\LARGE
		{\bf
		Отчет по лабораторной работе №7 \\ <<Прямая и обратная задача кинематики. DH-параметры>> \\по дисциплине <<Введение в профессиональную деятельность>>}\\
	\end{center}
	\begin{flushright}
		\Large
		\vspace*{10em plus 11em minus 9em}
		Выполнили студенты гр. {\textbf {R3135}}\\
		Трубицына А.М.\\
		Вьюгина А.П.\\
		Репин А.В.\\
		Макаренко К.С.\\
		Преподаватель: Перегудин А.А.,\\
		ассистент фак. СУиР		
	\end{flushright}
	\vspace*{15em plus 7em minus 6em}
	\begin{center}
		Санкт-Петербург 2020
	\end{center}
	\end{titlepage}


\section{Цель работы}
Ознакомиться со способом нахождения параметров манипулятора и научиться переходить из декартовых координат в обобщенные и обратно.

\section{Описание собранного робота}
\subsection*{Параметры Денавита-Харденберга}
\begin{tabular}{|l||c|c|c|c|}
\hline
Звено i & $a_i$ & $\alpha_i$ & $d_i$ & $\theta_i$\\
\hline
\hline
1 & 0,7 см & $\frac{\pi}2$ & 18 см & $\theta_1$  \\ \hline
2 & 12,8 см & 0 & 0 & $\theta_2$ + $\frac{\pi}2$ \\ \hline
3 & 12,8 см & 0 & 0 & $\theta_3$\\
\hline
\end{tabular}

\section{Коды для перехода из одной системы координат в другую}
\subsection{Функция для перехода из обобщенной СК в Декартову с помощью геометрических расчетов}
\subsubsection*{Вывод формул}
$x = \cos{\theta_1}(a_2\cos{\theta_2} + a_3\cos{\theta_3})$ \\
$y = \sin{\theta_1}(a_2\cos{\theta_2} + a_3\cos{\theta_3})$ \\
$z = d_1 + a_2\sin{\theta_2} + a_3\sin{\theta_3}$
\subsubsection*{Код}
\begin{minted}[linenos=true]{python}
def straight_transfer_formulas(theta1, theta2, theta3):
    theta3 = theta2-theta3
    theta1 = np.deg2rad(theta1)
    theta2 = np.deg2rad(theta2)
    theta3 = np.deg2rad(theta3)
    a = [0, 0.007, 0.128, 0.128]  # массив значений a_i
    alpha = [0, pi / 2, 0, 0]
    d = [0, 0.18, 0, 0]  # массив значений d_i
    x = (a[2] * cos(theta2) + a[3] * cos(theta3)) * cos(theta1)
    y = (a[2] * cos(theta2) + a[3] * cos(theta3)) * sin(theta1)
    z = d[1] + a[2] * sin(theta2) + a[3] * sin(theta3)
    print("x = ", x)
    print("y = ", y)
    print("z = ", z)

\end{minted}

\subsection{Функция для перехода из обощенной СК в Декартову с помощью матричных расчетов}
\subsubsection*{Вывод формул}

\subsubsection*{Код}
\begin{minted}[linenos=true, breaklines=true]{python}
def straight_transfer(theta1, theta2, theta3):
    theta3 = -theta3
    # перевод из градусов в радианы
    theta1 = np.deg2rad(theta1)
    theta2 = np.deg2rad(theta2)
    theta3 = np.deg2rad(theta3)
    Q = np.array([[0], [0], [0], [1]])

    a = [0, 0, 0.128, 0.128]  # массив значений a_i
    alpha = [0, pi / 2, 0, 0]
    #alpha = [0, pi / 2, 0, pi / 2]
    d = [0, 0.18, 0, 0]  # массив значений d_i

    T_0_1 = np.array([[cos(theta1), -sin(theta1) * cos(alpha[1]), sin(theta1) * sin(alpha[1]), a[1] * cos(theta1)],
                      [sin(theta1), cos(theta1) * cos(alpha[1]), -cos(theta1) * sin(alpha[1]), a[1] * sin(theta1)],
                      [0, sin(alpha[1]), cos(alpha[1]), d[1]],
                      [0, 0, 0, 1]])

    T_1_2 = np.array([[cos(theta2), -sin(theta2) * cos(alpha[2]), sin(theta2) * sin(alpha[2]), a[2] * cos(theta2)],
                      [sin(theta2), cos(theta2) * cos(alpha[2]), -cos(theta2) * sin(alpha[2]), a[2] * sin(theta2)],
                      [0, sin(alpha[2]), cos(alpha[2]), d[2]],
                      [0, 0, 0, 1]])

    T_2_3 = np.array([[cos(theta3), -sin(theta3) * cos(alpha[3]), sin(theta3) * sin(alpha[3]), a[3] * cos(theta3)],
                      [sin(theta3), cos(theta3) * cos(alpha[3]), -cos(theta3) * sin(alpha[3]), a[3] * sin(theta3)],
                      [0, sin(alpha[3]), cos(alpha[3]), d[3]],
                      [0, 0, 0, 1]])
    T = T_0_1.dot(T_1_2).dot(T_2_3)
    XYZ = T.dot(Q)
    print("x = ", XYZ[0][0])
    print("y = ", XYZ[1][0])
    print("z = ", XYZ[2][0])
\end{minted}

\subsection{Функция для перехода из Декартовой СК в обобщенную}
\begin{minted}[linenos=true]{python}
def backwards_transfer(x, y, z):
    a = [0, 0, 0.128, 0.128]  # массив значений a_i
    d = [0, 0.18, 0, 0]  # массив значений d_i
    r1 = sqrt(x ** 2 + y ** 2)
    r2 = z - d[1]
    r3 = sqrt(r1 ** 2 + r2 ** 2)
    
    theta1 = atan(y / x)
    psi1 = acos((a[2] ** 2 + r3 ** 2 - a[3] ** 2) / (2 * a[2] * r3))
    psi2 = atan(r2 / r1)
    theta2 = psi2 + psi1
    psi3 = acos((a[2] ** 2 + a[3] ** 2 - r3 ** 2) / (2 * a[2] * a[3]))
    theta3 = pi - psi3

    theta1 = np.rad2deg(theta1)
    theta2 = np.rad2deg(theta2)
    theta3 = np.rad2deg(theta3)

    print("theta1 = ", theta1)
    print("theta2 = ", theta2)
    print("theta3 = ", theta3)
\end{minted}

\section{Код для движения робота}
\subsection{Поочередный поворот звеньев манипулятора на заданные углы}
\begin{minted}[linenos=true, breaklines=true]{python}
from ev3dev.ev3 import *
import time


def saturate(x, left, right):
    if x > right: x = right
    if x < left: x = left
    return x


sound = Sound()
sound.set_volume(100)
sound.beep()

# первая координата - motorA
# вторая координата - motorB
# третья координата - motorC
q0 = [90, 45, 45]

# калибровка координат
q0 = [saturate(q0[0], -180, 180), saturate(q0[1], -70, 40), saturate(q0[2], -120, 100)]
q = [5 * q0[0], -5 * q0[1], -5/3 * q0[2]]

# значение коэффициентов в градусных мерах
k_p = [0.3, 0.3, 0.1]
k_i = [0.25/60, 0.25/60, 0]
k_d = [1/60, 1/60, 0]


motorA = LargeMotor('outA')
motorB = LargeMotor('outB')
motorC = MediumMotor('outC')

motorA.position = 0
motorB.position = 0
motorC.position = 0

timeStart = time.time()
last_t = time.time()
sum = 0
last_e = 0
inaccuracy = 5  # погрешность в градусах
U_max = 6.97

name = str(q0[0]) + "_" + str(q0[1]) + "_" + str(q0[2]) + ".txt"
file = open(name, 'w')

motors_set = [motorA, motorB, motorC]

for i in range(3):
    while abs(q[i] - motors_set[i].position) > inaccuracy:
        e = q[i] - motors_set[i].position
        dt = time.time() - last_t
        U = k_p[i] * e + k_d[i] * (e - last_e) / dt + k_i[i] * sum * dt
        U = U/U_max*100
        motors_set[i].run_direct(duty_cycle_sp=saturate(U, -100, 100))
        file.write(str(motorA.position) + '\t' + str(motorB.position) + '\t' + str(motorC.position) + '\t' + str(
            saturate(U, -100, 100)) + '\t' + str(k_p[i] * e) + '\t' + str(k_d[i] * (e - last_e) / dt) + '\t' +
                   str(k_i[i] * sum * dt) + '\n')
        sum += e
        last_e = e
        last_t = time.time()
    sum = 0
    last_e = 0
    last_t = time.time()
    motors_set[i].run_direct(duty_cycle_sp=0)

file.close()
\end{minted}

\section{Выводы}

\end{document}
